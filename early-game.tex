
\section{Early game conventions}
\label{sec:early-game}

We denote by \emph{early game} the phase of the game before the first blind discard. Misplaying a card does not end the early game. Discarding a clued card that is known to be useless does not end the early game either. As we already said, during the early game most of the relevant information involves \C{X1}'s and \C{X2}'s, so it would be a waste to explicitly give clues on that card. There are some useful rules that can make the first few clues extremely efficient.

\subsection{First clue}

Giving a 1-for-1 play clue as first clue is often inefficient, as it might be interfering with a better play clue that touches the first player's hand as well.

\begin{convention}[Non-intersection principle]
	Any \emph{play clue} given by the first player must either be at least 2-for-1, except for a 1-for-1 clue touching a card that can't possibly be in that player's hand (e.g. \C{K1} or any \C{X1} of which all three copies are visible).
\end{convention}

If the first player cannot give such a clue, they are allowed to clue any \C{X2} or \C{X5} without triggering a finesse.

\subsection{Discards}

It's highly unlikely that a player is allowed to discard during the early game, because there's a chance that no one had the possibility to tell them not to do so.

\begin{convention}[Early game discard]
	A player is not allowed to perform the first discard if all the players before them either played, gave \emph{play clues}, or gave \emph{fix clues} (including pre-emptive fix clues intended as setup for a follow-up play clue). These clues must be extinguished before giving any \emph{save clue}. A player is also not allowed to perform the first discard if they received the first \emph{save clue} of the game.
\end{convention}

There is an obvious exception: a player is allowed to give a \emph{save clue} (rather than a \emph{play clue} or a \emph{fix clue}) if not doing so would result in potentially discarding a card that needs to be saved, unless they can also prevent it by giving a \emph{play clue} instead.

Sometimes, players let a \C{X1} or a \C{X2} be discarded on purpose, because they appear in multiple copies or in multiple hands.

\begin{convention}[Early game discard notes]
	During the early game, if a \C{X1} or a \C{X2} is purposely not saved and happens to be discarded, then Convention~\ref{discard-notes} apply even if the double discard does not actually happen.
\end{convention}

\subsection{Playing order}

\begin{convention}[1's playing order]
	When a player receives a \C{X1} clue, they are supposed to play them starting from the ones that were not in the starting hand, starting from the chop and then from left to right, and then the ones that were in their starting hand, from right to left.
\end{convention}

This system allows for a better management of duplicated \C{X1}'s, in a way such that there is often no need of a \emph{fix clue} to prevent a misplay.

When multiple play clues are available, it is usually better to give them to the first player in turn order, as long as this does not lower the efficiency. If this doesn't happen, there must be a reason, usually that some of the \C{X1} that have been skipped are duplicated.

\begin{convention}[Skipped 1's]
	A \C{X1} clue touching two or more cards that were all in the starting hand, and that has been skipped in favour of another clue with lesser or equal efficiency given to a player coming after the one who just received the clue, implies that at least one of the clued cards is trash. The player who received the clue is supposed to play all but one of them from right to left and discard the last one, unless they receive other conflicting information.
\end{convention}

\subsection{Complements}

\begin{convention}[Complement 3 and 4]
	Any \C{X3} or \C{X4} clue given during the early game that is not a \emph{fix clue} is a \emph{complement clue} instead. The player who receives it is supposed to play all their other cards, starting from cards that already have a \emph{play clue} on them in the usual priority order (see Subsection~\ref{ssec:priorities}), if any, and following with all the remaining unclued cards, \emph{from right to left} (the opposite of the usual order).
\end{convention}

These clues can also be given by recluing a \C{X5}.

\begin{convention}[Complement double 5]
	Recluing a \C{X5} during the early game is also a \emph{complement clue}, and it is to be interpreted exactly as a \C{X3} or \C{X4} complement clue.
\end{convention}

Complement clues can trigger finesses as well, but sometimes this can get confusing if the clue requires a fix.

\begin{convention}[Two card promise and 5 fix]
	A complement clue leaving two or more unclued cards promises that at least two of them are playable (possibly through finesses), and that if there is a third unclued card that is not playable, then it must be a \C{X5}.
\end{convention}

This makes it clear when a card is included in a complement (possibly triggering finesses) and when the clue requires a fix instead.

Using complement clues this way gives some interesting corollaries.

\begin{convention}[All-but-one-1's save]
	If a player gets a \C{X1} clue on all but one unclued card during the early game, then the remaining card must be a \C{X2} or \C{X5}. That card is permanently chop moved (it cannot be discarded), and it counts as having a \C{X2} clue for the purpose of prompts and finesses only. If that card also has a negative \C{X2} or \C{X5}, then it counts as having the appropriate number clue for all purposes.
\end{convention}

Since a \C{X3} or \C{X4} clue given during the early game is a complement, in order to save a black \C{X3} or \C{X4} a colour clue is needed.

\begin{convention}[Early game black saves]
	During the early game, a \C{KX} clue on a chop card has to be interpreted as \emph{save clue} on that card, which must be a \C{K3} or a \C{K4}. As usual, \C{K2} and \C{K5} must be saved with a \emph{number clue}.
\end{convention}

Sometimes a \C{X3} or \C{X4} is misplayed during the early game, making saves difficult. If that happens, complements are off for that type of clue given on a chop card. The same holds from the beginning of the game (regardless of misplays) in any variant with a dark suit that does not allow for colour saves, such as \emph{dark rainbow} or \emph{grey}.

\begin{convention}[Early game misplay]
	If a \C{X3} is misplayed during the early game, then a \C{X3} clue on a chop card has to be interpreted as \emph{save clue} on that card, unless all the critical \C{X3}'s are visible by both the player giving the clue and the player receiving it. The same holds with \C{X4}'s.
\end{convention}

\begin{corollary}
	If a \C{X3} or \C{X4} clue on a chop card that can be interpreted as a save clue is instead interpreted as a complement clue, it means that all the critical \C{X3}'s or \C{X4}'s are visible by both the player giving the clue and the player receiving it, so if another player can't see one or more of them, these cards must be in their hand.
\end{corollary}

\subsection{Fix clues for complements}

It can happen that a complement clue must be given immediately, but it would trigger a finesse and the connecting card (that must be visible in some player's hand) is not in finesse position. In this case, the clue can still be given, and if the next card in the complement is not immediately playable, the other players should act as follows.

\begin{itemize}
	\item If, at the moment in which the clue was given, the connecting card was visible in any other player's hand (except the one who originally gave the clue), and it was in finesse position, they should assume that the original clue is implying a finesse on that card.
	\item If not, if they see a copy of the connecting card in the hand of any player strictly between them and the one who originally gave the clue, and that card was not in finesse position at the moment the clue was given, then they should give a fix clue.
	\item If not, if they see a copy of the connecting card in the hand of any player strictly between the one who originally gave the clue and the one who received the clue, and that card was not in finesse position at the moment the clue was given, then they may clue that card (but don't have to).
	\item If they can't see any copy of the connecting card, then they should assume that the original clue was a finesse, hence they should play their card that was in finesse position when the clue was given.
\end{itemize}

The same holds for a card in the complement that is 2-away or more, even if it is not the text one to be played, (remember that it is promised to be playable unless it is a \C{X5}) if any connecting card must be played before the original cluer's next turn to prevent future misplays.