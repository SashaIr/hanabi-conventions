\section{Prompts and finesses}

Clued cards should be played as soon as possible. Following this principle, one can give important information by cluing cards that are not immediately playable, by implying that they will be playable soon. This can be summed up in the following principle.

\begin{convention}[Connection principle]
	\label{connection-principle}
	If a card that is not immediately playable is given a play clue, then all the connecting cards whose position is not currently known must be visible by the player who gave the clue.
\end{convention}

By \emph{connecting cards} we mean all the cards that have to be played before the clued one (e.g. if the \C{B2} is on the board and the \C{B4} is clued, then the connecting card is the \C{B3}). Such a clue \emph{promises} all the connecting cards.

\subsection{Prompt}

The easiest of these conventions is the \emph{prompt}, which involve only already clued cards.

\begin{convention}[Prompt]
	\label{prompt}
	If a player deduces that they have a connecting card, then it must be the leftmost among the clued ones that can be that card.
\end{convention}

\begin{remark}
	Some options for a card might be ruled out by \emph{save notes} (see Convention~\ref{save-notes}) or \emph{double discard notes} (see Convention~\ref{discard-notes}). In these cases, that card does \emph{not} count as compatible.
\end{remark}

\begin{example} \hfill \\
	\begin{minipage}{0.45\textwidth}
		\begin{itemize}
			\item[\Large +] \CARD{R1} \CARD{Y1} \CARD{G3} \CARD{BX} \CARD{P2} \CARD{KX}
			\item[\Large A] \CARD[c]{B1} \CARD{G2} \CARD[cn]{B4} \CARD[n]{R5}
			\item[\Large B] \CARD[n]{R2} \CARD{Y1} \CARD[n]{P5} \CARD[n]{K2}
			\item[\Large C] \CARD{Y4} \CARD{P4} \CARD{B3} \CARD[c]{K4}
			\item[\Large D] \CARD{B5} \CARD{R3} \CARD{R4} \CARD{Y4}
			\item[\Large E] \CARD[c]{P3} \CARD[c]{P4} \CARD{G3} \CARD{Y1}
		\end{itemize}
	\end{minipage}%
	\begin{minipage}{0.55\textwidth}
		\hfill \\
		
		\textbf{Description.} \\
		
		C to play. If they clue \C{RX} to D, then this is a play clue. However, according to Convention~\ref{disjoint-clues}, none of their cards is a \C{R2}, since it has already been clued in B's hand (probably with a \C{X2}-save, since they have \C{K2} on chop). Hence, D doesn't play. Then, when B is to play, since they know that a play clue has been given on \C{R3}, and they can't see the \C{R2} in any other players hand, they can deduce that the \C{R2} is in their hand. According to Convention~\ref{prompt}, it must be the leftmost of the two \C{X2}'s, so they can safely play it.
	\end{minipage}
\end{example} \vspace{0.15 cm}

Players are allowed to lie if this gets more cards played, as in the next example.

\begin{example} \hfill \\
	\begin{minipage}{0.45\textwidth}
		\begin{itemize}
			\item[\Large +] \CARD{R1} \CARD{Y1} \CARD{G3} \CARD{BX} \CARD{P2} \CARD{KX}
			\item[\Large A] \CARD[c]{B1} \CARD{G2} \CARD[n]{B4} \CARD[n]{R5}
			\item[\Large B] \CARD[n]{Y2} \CARD{Y1} \CARD[n]{P5} \CARD[n]{R2}
			\item[\Large C] \CARD{Y4} \CARD{P4} \CARD{B3} \CARD[c]{K4}
			\item[\Large D] \CARD{B5} \CARD{R3} \CARD{R4} \CARD{Y4}
			\item[\Large E] \CARD[c]{P3} \CARD[c]{P4} \CARD{G3} \CARD{Y1}
		\end{itemize}
	\end{minipage}%
	\begin{minipage}{0.55\textwidth}
		\hfill \\
		
		\textbf{Description.} \\
		
		Same example as before, but the \C{X2}'s in B's hand are now of different colours. In particular, the leftmost is a \C{Y2}, not a \C{R2}. However, this is not a problem, since the \C{Y2} is playable: during their turn, B will deduce that their leftmost \C{X2} is \C{R2}, and they will play it. One round later, since B knows that they must have the \C{R2} in their hand, they will play their other clued \C{X2}, which this time would be the correct one. B was tricked into think that their leftmost \C{X2} was \C{R2} in order to get it played for free.
	\end{minipage}
\end{example} \vspace{0.15 cm}

\subsection{Finesse}
\label{sec:finesse}

The finesse is probably the most important convention in Hanabi. It takes a while to get used to it, but then it's an extremely powerful tool. It is the same as a \emph{prompt}, except that the connecting card is not clued. The key to this convention is the position of the card.

\begin{convention}[Finesse]
	\label{finesse}
	If a player deduces that they have a connecting card, and none of their clued cards (if any) is compatible, then it must be the leftmost among the unclued ones.
\end{convention}

\begin{remark}
	As for the \emph{prompt}, some options for a card might be ruled out by \emph{save notes} (see Convention~\ref{save-notes}) or \emph{double discard notes} (see Convention~\ref{discard-notes}). In these cases, that card does \emph{not} count as compatible.
\end{remark}

\begin{example} \hfill \\
	\begin{minipage}{0.45\textwidth}
		\begin{itemize}
			\item[\Large +] \CARD{R1} \CARD{Y2} \CARD{G5} \CARD{B3} \CARD{P2} \CARD{KX}
			\item[\Large A] \CARD{B1} \CARD{G2} \CARD[n]{R5} \CARD[n]{B4}
			\item[\Large B] \CARD{P1} \CARD{Y2} \CARD[n]{K2} \CARD[n]{P5}
			\item[\Large C] \CARD{Y3} \CARD{P4} \CARD{R3} \CARD[c]{K4}
			\item[\Large D] \CARD{B2} \CARD{G4} \CARD{Y4} \CARD{R1}
			\item[\Large E] \CARD[c]{P3} \CARD[c]{P4} \CARD{G3} \CARD{Y1}
		\end{itemize}
	\end{minipage}%
	\begin{minipage}{0.55\textwidth}
		\hfill \\
		
		\textbf{Description.} \\
		
		If B is to play, they can clue \C{YX} to D. C sees that D is given a play clue on their non-playable \C{Y4}, and also they can't see the connecting \C{Y3} in any other player's hand. Hence, C must have the \C{Y3}, and since their only clued card is the \C{K4} (which, being \C{KX}, can't be \C{YX}), then they should deduce that the connecting card is their leftmost, and play it.
	\end{minipage}
\end{example} \vspace{0.15 cm}

This trick gets two cards played with one clue, and hence it is a very powerful way to pass information.

\subsection{The reverse finesse}

The finesse works even if the player who gets the play clue comes before the one that is supposed to have the connecting card. 

\begin{convention}[Reverse finesse]
	If a player who is given a play clue sees that some other player has, as their leftmost unclued card, another card that matches the clue they just received, they should wait at least one round before playing.
\end{convention}

If the player who has the matching card as leftmost unclued one plays it, then it means that the clued card is its successor. If they don't, then it is the same one. In any case the clued card should be played the next round.

\begin{example} \hfill \\
	\begin{minipage}{0.45\textwidth}
		\begin{itemize}
			\item[\Large +] \CARD{R1} \CARD{Y2} \CARD{G5} \CARD{B3} \CARD{P2} \CARD{KX}
			\item[\Large A] \CARD{B1} \CARD{G2} \CARD[n]{R5} \CARD[n]{B4}
			\item[\Large B] \CARD{P1} \CARD{Y2} \CARD[n]{K2} \CARD[n]{P5}
			\item[\Large C] \CARD{B2} \CARD{G4} \CARD{Y4} \CARD{R1}
			\item[\Large D] \CARD{Y3} \CARD{P4} \CARD{R3} \CARD[c]{K4}
			\item[\Large E] \CARD[c]{P3} \CARD[c]{P4} \CARD{G3} \CARD{Y1}
		\end{itemize}
	\end{minipage}%
	\begin{minipage}{0.55\textwidth}
		\hfill \\
		
		\textbf{Description.} \\
		
		Same as before, except that C and D have been swapped. If B is to play, they can clue \C{YX} to C. C sees that D has a \C{Y3} as drop, hence the clued card might be a \C{Y4} and C does not play it. Then, D should deduce that they have a \C{Y3} and play their leftmost card, as before.
	\end{minipage}
\end{example} \vspace{0.15 cm}

\begin{remark}
	What if we replace the \C{Y4} with a \C{Y3}? In fact, this is no issue at all: after C's turn, D just doesn't play. In fact, maybe C had another good reason to not play the yellow-clued card. Even if D can deduce to have a \C{Y3} as leftmost unclued card, they should not play it, else C would think that their card is a \C{Y4}.
	
	There are other better ways to deal with it. B can clue \C{YX} to D instead: there is no other \C{Y3} in anyone's slot 1, so they will know they have a \C{Y3}. Alternatively, B can simply not clue anything to C, and instead let C clue \C{YX} to D. Even better, if B happens to have a \C{Y4} in their hand (and they can't know), C could clue \C{YX} to them instead, saving one clue.
\end{remark}

\subsection{Layered finesse}

The finesse can be used in a lot more cases, possibly combined with prompts as well, to get extra cards played.

\begin{definition}
	\label{def:finesse-position}
	A card is in \emph{finesse position} for a clue if the card is promised by the clue, and one of the following holds:
	
	\begin{itemize}
%		\item it is clued, the clue is compatible with the promised card, and it is the leftmost among the cards with these properties;
		\item it is clued, the clue is compatible with the promised card, and all the cards with these properties on its left are playable;
%		\item it is unclued, it is the leftmost among the unclued cards, and all the clued compatible cards in that player's hand (if any) are playable;
		\item it is unclued, all the clued compatible cards in that player's hand (if any) are playable, and all the unclued cards on its left are playable.
	\end{itemize}
\end{definition}

\begin{example} \hfill \\
	\begin{minipage}{0.45\textwidth}
		\begin{itemize}
			\item[\Large +] \CARD{R1} \CARD{Y2} \CARD{G5} \CARD{B3} \CARD{P2} \CARD{KX}
			\item[\Large A] \CARD{B1} \CARD{G2} \CARD[n]{R5} \CARD[n]{B4}
			\item[\Large B] \CARD{P1} \CARD{Y2} \CARD[n]{K2} \CARD[n]{P5}
			\item[\Large C] \CARD{R2} \CARD{R3} \CARD{Y3} \CARD[c]{K4}
			\item[\Large D] \CARD{B2} \CARD{G4} \CARD{Y4} \CARD{R1}
			\item[\Large E] \CARD[c]{P3} \CARD[c]{P4} \CARD{G3} \CARD{Y1}
		\end{itemize}
	\end{minipage}%
	\begin{minipage}{0.55\textwidth}
		\hfill \\
		
		\textbf{Description.} \\
		
		If A clues \C{YX} to D, they are promising a \C{Y3}. C has a \C{Y3}, and it is in finesse position: it is unclued, there are no compatible clued cards, and all the cards on its left will be playable at the appropriate moment. However, if C also got a \C{X3} clue, then the \C{Y3} would not be in finesse position any more! In fact, it would be clued with a compatible clue, but in that case the \C{R3} would also be, it is on the left of the \C{Y3}, and it is not playable.
	\end{minipage}
\end{example} \vspace{0.15 cm}

This example explains why it is not important that the promised card is exactly where it is expected to be, as long as it is in finesse position. This leads to the following.

\begin{convention}[Layered finesse]
	\label{layered-finesse}
	The finesse applies as long as the connecting card is in finesse position.
\end{convention}

\begin{example} \hfill \\
	\begin{minipage}{0.45\textwidth}
		\begin{itemize}
			\item[\Large +] \CARD{R1} \CARD{Y2} \CARD{G5} \CARD{B3} \CARD{P2} \CARD{KX}
			\item[\Large A] \CARD{B1} \CARD{G2} \CARD[n]{R5} \CARD[n]{B4}
			\item[\Large B] \CARD{P1} \CARD{Y2} \CARD[n]{K2} \CARD[n]{P5}
			\item[\Large C] \CARD{R2} \CARD{Y3} \CARD{R3} \CARD[c]{K4}
			\item[\Large D] \CARD{B2} \CARD{G4} \CARD{Y4} \CARD{R1}
			\item[\Large E] \CARD[c]{P3} \CARD[c]{P4} \CARD{G3} \CARD{Y1}
		\end{itemize}
	\end{minipage}%
	\begin{minipage}{0.55\textwidth}
		\hfill \\
		
		\textbf{Description.} \\
		
		The setting is the same as Subsection~\ref{sec:finesse}, except that now C's leftmost unclued card is a \C{R2}, which is playable. The next one is a \C{Y3}. If A is to play, they can still clue \C{YX} to D. As before, C should play their leftmost unclued card, and so they do. Since C played, D can deduce that their card is not a \C{Y3}, but it's a \C{Y4} instead, so they wait. During their next turn C should keep playing, and so they have to pick their second leftmost unclued card \emph{at the moment they received the clue} (which is a very important information to track). They play the \C{Y3} and next D plays the \C{Y4}, leading to play three cards with one clue.
	\end{minipage}
\end{example} \vspace{0.15 cm}

\begin{remark}
	A \emph{finessed} player should keep playing until they see the expected finessed card, or they get a stop sign (which will be discussed later). The cards should be played in the order given by Definition~\ref{def:finesse-position}, and the first one in that order will be referred to as \emph{drop} from now on.
\end{remark}

\subsection{Multiplayer finesse}

\begin{convention}[Multiplayer finesse]
	\label{multiplayer-finesse}
	If all the connecting cards are in finesse position but spread among multiple players, the finesse still applies.
\end{convention}

\begin{example} \hfill \\
	\begin{minipage}{0.45\textwidth}
		\begin{itemize}
			\item[\Large +] \CARD{R1} \CARD{Y1} \CARD{G5} \CARD{B3} \CARD{P2} \CARD{KX}
			\item[\Large A] \CARD{B1} \CARD{G2} \CARD[n]{R5} \CARD[n]{B4}
			\item[\Large B] \CARD{Y2} \CARD{P2} \CARD{K2} \CARD[n]{P5}
			\item[\Large C] \CARD{Y3} \CARD{R3} \CARD{B3} \CARD[c]{K4}
			\item[\Large D] \CARD{B2} \CARD{Y1} \CARD{Y4} \CARD{R1}
			\item[\Large E] \CARD[c]{K1} \CARD{P4} \CARD{G3} \CARD{Y1}
		\end{itemize}
	\end{minipage}%
	\begin{minipage}{0.55\textwidth}
		\hfill \\
		
		\textbf{Description.} \\
		
		A clues \C{X4} to D. It is a finesse on the \C{Y4} (see Convention~\ref{fake-early-save}), so B is supposed to play their drop card, which is the \C{Y2}. C sees that the \C{Y4} still lacks a connecting card (the \C{Y3}) that they can't see in any other player's hand. Hence, C is also supposed to play their drop card.
	\end{minipage}
\end{example} \vspace{0.15 cm}

\subsection{Bluff}

An exception to the layered finesse convention is the \emph{bluff}. A player can be tricked into thinking that they have the connecting card even if they don't, just to get their drop played. This might lead into confusion, since all the other players will assume that they have the connecting card, so a precise criterion to distinguish bluffs from multiple finesses is needed.

\begin{convention}[Bluff]
	\label{bluff}
	If a player gives a \emph{play clue} focusing on a one-away card (i.e. a card that needs only one connecting card) of which the connecting card is missing, then the very next player has to play their drop card and not continue playing into the finesse during their next turn.
	
	Everyone should mark that card as one-away (in particular, if the play clue is a colour clue, the card is considered completely clued for all purposes). The player who got the clue should always assume a finesse over a bluff if possible. No one should assume that the connecting card is visible in any player's hand.
\end{convention}

This might be quite confusing, so some examples are needed.

\begin{example}
	\label{ex:bluff}
	\hfill \\
	\begin{minipage}{0.45\textwidth}
		\begin{itemize}
			\item[\Large +] \CARD{R1} \CARD{Y2} \CARD{G5} \CARD{B3} \CARD{P2} \CARD{KX}
			\item[\Large A] \CARD{B1} \CARD{G2} \CARD[n]{R5} \CARD[n]{B4}
			\item[\Large B] \CARD{P1} \CARD{Y2} \CARD[n]{K2} \CARD[n]{P5}
			\item[\Large C] \CARD{R2} \CARD{P4} \CARD{R3} \CARD[c]{K4}
			\item[\Large D] \CARD{B2} \CARD{G4} \CARD{Y4} \CARD{R1}
			\item[\Large E] \CARD[c]{P3} \CARD[c]{P4} \CARD{G3} \CARD{Y1}
		\end{itemize}
	\end{minipage}%
	\begin{minipage}{0.55\textwidth}
		\hfill \\
		
		\textbf{Description.} \\
		
		B clues \C{YX} to D. As we've seen before, C is supposed to play their drop, which is the \C{R2}. Since C blind-played his newest card, D should deduce that its yellow card is one-away, and hence it is a \C{Y4}. Since the \C{R2} can't be connecting, D shouldn't play their yellow card; since the \C{Y4} is one-away, all the other players should deduce that they do not have a \C{Y3} and so they should not play their drop card.
	\end{minipage}
\end{example} \vspace{0.15 cm}

\begin{example}	\hfill \\
	\begin{minipage}{0.45\textwidth}
		\begin{itemize}
			\item[\Large +] \CARD{R1} \CARD{Y2} \CARD{G5} \CARD{B3} \CARD{P2} \CARD{KX}
			\item[\Large A] \CARD{B1} \CARD{G2} \CARD[n]{R5} \CARD[n]{B4}
			\item[\Large B] \CARD{P1} \CARD{Y2} \CARD[n]{K2} \CARD[n]{P5}
			\item[\Large C] \CARD{Y3} \CARD{P4} \CARD{R3} \CARD[c]{K4}
			\item[\Large D] \CARD{B2} \CARD{G4} \CARD{Y4} \CARD{R1}
			\item[\Large E] \CARD[c]{P3} \CARD[c]{P4} \CARD{G3} \CARD{Y1}
		\end{itemize}
	\end{minipage}%
	\begin{minipage}{0.55\textwidth}
		\hfill \\
		
		\textbf{Description.} \\
		
		B clues \C{YX} to D. As we've seen before, C is supposed to play their drop card, which is the \C{Y3}. Since C blind-played his newest card, D should deduce that its yellow card is one-away, and hence it is a \C{Y4}. Since the \C{Y3} can be connecting, D should play their yellow card.
	\end{minipage}
\end{example} \vspace{0.15 cm}

\begin{example}	\hfill \\
	\begin{minipage}{0.45\textwidth}
		\begin{itemize}
			\item[\Large +] \CARD{R1} \CARD{Y2} \CARD{G5} \CARD{B3} \CARD{P2} \CARD{KX}
			\item[\Large A] \CARD{B1} \CARD{G2} \CARD[n]{R5} \CARD[n]{B4}
			\item[\Large B] \CARD{P1} \CARD{Y2} \CARD[n]{K2} \CARD[n]{P5}
			\item[\Large C] \CARD{Y3} \CARD{G4} \CARD{R3} \CARD[c]{K4}
			\item[\Large D] \CARD{B2} \CARD{Y1} \CARD{Y4} \CARD{R1}
			\item[\Large E] \CARD[c]{P3} \CARD[c]{P4} \CARD{G3} \CARD{Y1}
		\end{itemize}
	\end{minipage}%
	\begin{minipage}{0.55\textwidth}
		\hfill \\
		
		\textbf{Description.} \\
		
		In this case, cluing \C{YX} doesn't work because of the \C{Y1}, so B clues \C{X4} to D. The \C{Y4} is one-away, so C is supposed to play their drop, which is the \C{Y3}. D deduces that their \C{X4} is one-away, so it might be either yellow or white, but since the \C{Y3} can be connecting, D should play their \C{X4} anyway.
	\end{minipage}
\end{example} \vspace{0.15 cm}

\begin{example}	\hfill \\
	\begin{minipage}{0.45\textwidth}
		\begin{itemize}
			\item[\Large +] \CARD{R1} \CARD{Y2} \CARD{G5} \CARD{B3} \CARD{P2} \CARD{KX}
			\item[\Large A] \CARD{B1} \CARD{G2} \CARD[n]{R5} \CARD[n]{B4}
			\item[\Large B] \CARD{P1} \CARD{Y2} \CARD[n]{K2} \CARD[n]{P5}
			\item[\Large C] \CARD{R2} \CARD{R3} \CARD{Y3} \CARD[c]{K4}
			\item[\Large D] \CARD{B2} \CARD{G4} \CARD{Y4} \CARD{R1}
			\item[\Large E] \CARD[c]{P3} \CARD[c]{P4} \CARD{G3} \CARD{Y1}
		\end{itemize}
	\end{minipage}%
	\begin{minipage}{0.55\textwidth}
		\hfill \\
		
		\textbf{Description.} \\
		
		Now it is A that clues \C{YX} to D. B sees that the \C{Y3} in C's hand only has playable cards on its left, hence, according to Convention~\ref{layered-finesse}, they deduce that C is the one that is supposed to play their drop. C plays it, and it is a \C{R2}. Since C blind-played his newest card, D should deduce that its yellow card is not a \C{Y3}, and hence it is probably a \C{Y4}. Since C was not the player immediately after the one who gave the play clue, they should keep playing until they see the \C{Y3}.
	\end{minipage}
\end{example} \vspace{0.15 cm}

\begin{example}	\hfill \\
	\begin{minipage}{0.45\textwidth}
		\begin{itemize}
			\item[\Large +] \CARD{R1} \CARD{Y1} \CARD{G5} \CARD{B3} \CARD{P2} \CARD{KX}
			\item[\Large A] \CARD{B1} \CARD{G2} \CARD[n]{R5} \CARD[n]{B4}
			\item[\Large B] \CARD{P1} \CARD{Y2} \CARD[n]{K2} \CARD[n]{P5}
			\item[\Large C] \CARD{Y2} \CARD{Y3} \CARD{R3} \CARD[c]{K4}
			\item[\Large D] \CARD{B2} \CARD{G4} \CARD{Y4} \CARD{R1}
			\item[\Large E] \CARD[c]{P3} \CARD[c]{P4} \CARD{G3} \CARD{Y1}
		\end{itemize}
	\end{minipage}%
	\begin{minipage}{0.55\textwidth}
		\hfill \\
		
		\textbf{Description.} \\
		
		Now is again B that clues \C{YX} to D. C is supposed to play their drop, which is the \C{Y2}. D sees that C also has a \C{Y3} on the right of the \C{Y2}, so their card might be the \C{Y4}. Since the \C{Y4} was not one-away, C should keep playing until they see the \C{Y3}.
	\end{minipage}
\end{example} \vspace{0.15 cm}

\begin{example}	\hfill \\
	\begin{minipage}{0.45\textwidth}
		\begin{itemize}
			\item[\Large +] \CARD{R1} \CARD{Y1} \CARD{G5} \CARD{B3} \CARD{P2} \CARD{KX}
			\item[\Large A] \CARD{B1} \CARD{G2} \CARD[n]{R5} \CARD[n]{B4}
			\item[\Large B] \CARD{Y2} \CARD{P2} \CARD[n]{K2} \CARD[n]{P5}
			\item[\Large C] \CARD{R2} \CARD{Y3} \CARD{B3} \CARD[c]{K4}
			\item[\Large D] \CARD{B2} \CARD{Y1} \CARD{Y4} \CARD{R1}
			\item[\Large E] \CARD[c]{K1} \CARD{P4} \CARD{G3} \CARD{Y1}
		\end{itemize}
	\end{minipage}%
	\begin{minipage}{0.55\textwidth}
		\hfill \\
		
		\textbf{Description.} \\
		
		A clues \C{X4} to D. It is a finesse on the \C{Y4}, so B is supposed to play their drop, which is the \C{Y2}. C sees that the \C{Y4} was not one-away (it was not the \C{P4}), hence it can't be a bluff, and according to Convention~\ref{multiplayer-finesse}, they should play their drop, the \C{R2}. D sees that C played, so their \C{X4} cannot be the \C{P4} (two people played, so it is not a bluff), but it must be the \C{Y4} (and it is a finesse). The next turn C will play the \C{Y3} (they know it is not a bluff, since the \C{Y4} is not one-away), and D will play their \C{Y4}.
	\end{minipage}
\end{example} \vspace{0.15 cm}

\begin{example}	\hfill \\
	\begin{minipage}{0.45\textwidth}
		\begin{itemize}
			\item[\Large +] \CARD{R1} \CARD{Y2} \CARD{G5} \CARD{B3} \CARD{P2} \CARD{KX}
			\item[\Large A] \CARD{B1} \CARD{G2} \CARD[n]{R5} \CARD[n]{B4}
			\item[\Large B] \CARD{P1} \CARD{Y2} \CARD[n]{K2} \CARD[n]{P5}
			\item[\Large C] \CARD{R2} \CARD{P4} \CARD{R3} \CARD[c]{K4}
			\item[\Large D] \CARD{B2} \CARD{G4} \CARD{Y5} \CARD{Y4}
			\item[\Large E] \CARD[c]{P3} \CARD[c]{P4} \CARD{G3} \CARD{Y1}
		\end{itemize}
	\end{minipage}%
	\begin{minipage}{0.55\textwidth}
		\hfill \\
		
		\textbf{Description.} \\
		
		This is the same as Example~\ref{ex:bluff}, except that the \C{YX} clue B gives to D now touches two cards. Because of Convention~\ref{chop-focus}, the focus of the clue is the card in discard position, i.e. the \C{Y4}. This still works as normal. Considering that C played, then D's rightmost card must be a \C{Y4}, and because of Convention~\ref{disjoint-clues} the other one is most likely (but not certainly) a \C{Y5}.
	\end{minipage}
\end{example} \vspace{0.15 cm}

\subsection{False bluff}

It is usually impossible to perform a multiple finesse on the very next player, as they will mistake it as a bluff. However, if the bluffed card could have been gotten by a regular finesse, then it doesn't make sense to get it through a bluff, and thus the clue must be a multiple finesse instead.

\begin{example}	\hfill \\
	\begin{minipage}{0.45\textwidth}
		\begin{itemize}
			\item[\Large +] \CARD{R2} \CARD{YX} \CARD{GX} \CARD{B1} \CARD{P3} \CARD{KX}
			\item[\Large A] \CARD{P3} \CARD{P4} \CARD{G4} \CARD{G3}
			\item[\Large B] \CARD{G1} \CARD[n]{P4} \CARD{R3} \CARD{B3}
			\item[\Large C] \CARD{P1} \CARD{B4} \CARD{G2} \CARD[c]{Y2}
			\item[\Large D] \CARD{Y5} \CARD{P1} \CARD{B3} \CARD{R4}
			\item[\Large E] \CARD{K1} \CARD{Y1} \CARD[n]{Y4} \CARD[n]{K2}
		\end{itemize}
	\end{minipage}%
	\begin{minipage}{0.55\textwidth}
		\hfill \\
		
		\textbf{Description.} \\
		
		Here, A clues \C{RX} to D. B plays their leftmost unclued card, which is a \C{G1}, and sees that C has a \C{G2}. It makes no sense for A to perform a \textit{bluff} here, as they can just clue \C{GX} to C and perform a finesse instead, which would be better than a bluff as the \C{G2} gets played too. It follows that the \C{RX} clue cannot be a bluff, and must be a \emph{layered finesse} instead. The next round, B will go on playing their \C{R3} (they skip \C{P4} as it has a \C{X4} clue on it), and D will then play their \C{R4}.
	\end{minipage}
\end{example} \vspace{0.15 cm}

\subsection{Discard finesse}

\begin{convention}[Discard finesse]
	If a player has full knowledge on a playable card (or a card that will be playable soon), and they discard it, it means that a player has another copy of the same card in finesse position. The player that cannot see the other copy of the card should hence safely play their drop card.
\end{convention}

Here, the definition of \emph{full knowledge} depends on the context; the player who discards must have strong reasons to believe that they know both colour and number of the card that they are discarding, and that the other players are aware of that.

This manoeuvre is usually not recommended (just playing the card is simpler), but it can be done in several contexts to gain an advantage. For example, it can be useful if the other copy of the card is clued (thus, discarding it prevents misplays); if it is in finesse position but it is not the leftmost unclued card (in order to get some extra card played for free); if the player who has the other copy is short on clues and/or they have a dangerous chop (keeping them busy and preventing them from discarding); if the card is not immediately playable but it will be during the next turn of the player that has the other copy (to gain tempo).

\subsection{Self finesse}

If there is no risk of misunderstanding, a finesse can be performed even if the connecting card is in the hand of the player who receives the clue. Such a finesse must be performed with a \emph{number clue}, as any colour clue would just be a \emph{play clue} on the clued cards.

\begin{convention}[Self finesse]
	A number clue on a card that is at least two-away from chop is a \emph{play clue}, and it originates a self finesse if no other interpretation is possible.
\end{convention}

\begin{example}	\hfill \\
	\begin{minipage}{0.45\textwidth}
		\begin{itemize}
			\item[\Large +] \CARD{R1} \CARD{Y2} \CARD{G5} \CARD{B3} \CARD{P2} \CARD{KX}
			\item[\Large A] \CARD{B1} \CARD{G2} \CARD[n]{R5} \CARD[n]{B4}
			\item[\Large B] \CARD{P1} \CARD{Y2} \CARD[n]{K2} \CARD[n]{P5}
			\item[\Large C] \CARD{R3} \CARD{P4} \CARD{G1} \CARD{K4}
			\item[\Large D] \CARD{B2} \CARD{G4} \CARD{Y5} \CARD{Y4}
			\item[\Large E] \CARD[n]{P4} \CARD{P3} \CARD{G3} \CARD{Y1}
		\end{itemize}
	\end{minipage}%
	\begin{minipage}{0.55\textwidth}
		\hfill \\
		
		\textbf{Description.} \\
		
		D just clued \C{X4} to E, who now is to play. It can't be a play clue on \C{B4}, as it is already clued in A's hand, so it must be another relevant \C{X4}, either \C{R4}, \C{Y4}, or \C{P4} (C has \C{K4}). The corresponding possible connecting cards are \C{R2}, \C{Y3}, and \C{P3}, none of which is visible, so it must be a self finesse. E plays their new finesse position card as either of those. It is \C{P3}, so they will assume a finesse on \C{P4} and they will go on playing their \C{X4} as \C{P4} on the next round. Notice that, since E didn't play the \C{X4} immediately, there must have been \C{B4} already clued in someone's hand. A knows this, deduces that their \C{X4} is in fact a \C{B4}, and plays it the very next turn.
	\end{minipage}
\end{example} \vspace{0.15 cm}

\subsection{Patch finesse}

The patch finesse is a complicated technique that can be used to achieve two or more separate finesses with one single clue.

\begin{convention}[Patch finesse]
	\label{patch-finesse}
	If a play clue on a non-playable card is given, and the connecting card is visible in some other player's hand, with another non-playable card on their left that is exactly one-away from playable, then the clue is also a finesse on that card.
\end{convention}

Let's see an example.

\begin{example}	\hfill \\
	\begin{minipage}{0.45\textwidth}
		\begin{itemize}
			\item[\Large +] \CARD{R1} \CARD{Y2} \CARD{G5} \CARD{B3} \CARD{P2} \CARD{KX}
			\item[\Large A] \CARD{R3} \CARD{Y3} \CARD[n]{R5} \CARD[n]{B4}
			\item[\Large B] \CARD{P1} \CARD{Y2} \CARD[n]{K2} \CARD[n]{P5}
			\item[\Large C] \CARD[c]{P3} \CARD[c]{P4} \CARD{G3} \CARD{Y1}
			\item[\Large D] \CARD{R2} \CARD{Y5} \CARD{Y1} \CARD[n]{K5}
			\item[\Large E] \CARD{B2} \CARD{G4} \CARD{Y4} \CARD{R1}
		\end{itemize}
	\end{minipage}%
	\begin{minipage}{0.55\textwidth}
		\hfill \\
		
		\textbf{Description.} \\
		
		B just clued \C{YX} to E, C plays their \C{P3}, and D is now to play. They see that A has a \C{Y3} in their hand, but not in finesse position. Cluing \C{RX} to A may be ambiguous, since A's red cards cannot be played. How can D deduce if they're being finessed or not? They can't, but it's not an issue at all: D should just play their leftmost unclued card. If that's a \C{Y3}, then D was being finessed everything is fine. If not, then A is being finessed, but their leftmost unclued card is a \C{R3}, not playable. Then there is also an ongoing finesse on that card and D's leftmost unclued card must be a \C{R2}, so they should still play.	
	\end{minipage}
\end{example} \vspace{0.15 cm}

\begin{remark}
	Patch finesses are very risky, and it require great understanding and trust among the players. It's very easy to mistake a generic clue for a patch finesse, so be careful and think to every possible scenario before giving such a clue. I recommend beginners not to use this convention until they are fully familiar with the finesse and with the play style of the other players.
\end{remark}

\subsection{Fixing a finesse}

\begin{convention}[Fixing a finesse]
	\label{unfinessing}
	If a play clue on a non-playable card is given, and the connecting card is visible in some other player's hand, with another non-playable card on their left that is not one-away, then it is still possible to perform a finesse, but that card must be clued before it leads to misplays.
\end{convention}

\begin{example}	\hfill \\
	\begin{minipage}{0.45\textwidth}
		\begin{itemize}
			\item[\Large +] \CARD{R1} \CARD{Y2} \CARD{G5} \CARD{B3} \CARD{P2} \CARD{KX}
			\item[\Large A] \CARD{B4} \CARD{K3} \CARD{Y3} \CARD[n]{R5}
			\item[\Large B] \CARD{P1} \CARD{Y2} \CARD[n]{K2} \CARD[n]{P5}
			\item[\Large C] \CARD{R2} \CARD{Y5} \CARD{Y1} \CARD[n]{K5} 
			\item[\Large D] \CARD{B2} \CARD{G4} \CARD{Y4} \CARD{R1}
			\item[\Large E] \CARD[c]{P3} \CARD[c]{P4} \CARD{G3} \CARD{Y1}
		\end{itemize}
	\end{minipage}%
	\begin{minipage}{0.55\textwidth}
		\hfill \\
		
		\textbf{Description.} \\
		
		Same as before, except for A's hand. B clues \C{YX} to D, trying to get the \C{B4} and the \C{Y3} for free. All the players but A see the connecting card in A's hand, and they also see that A's drop is a \C{B4}, which is playable, so nobody does anything. A plays their \C{B4} assuming that it is a \C{Y3} instead, and now B must stop A from playing their \C{K3}, with either a \C{KX} or a \C{X3} clue on that card. This can't possibly be a play clue (if it were a \C{K1}, then no clue would have been needed), nor a finesse on some other player (otherwise B could just have let that going, since everybody knew that A was about to play their \C{K3}), hence it must just be a \emph{fix clue}, meaning that the clued card is not involved into the finesse, and hence the card on its right has to be played next.
	\end{minipage}
\end{example} \vspace{0.15 cm}

The player that fixes a finesse should be the same one who performed the finesse before (because the clue might be mistaken for a patch finesse), but not always. In some cases there is no chance for the clue to be mistaken for a patch finesse (for example, if the \C{K3} is replaced by a \C{B1}), hence any player can give the fix clue. Distinguishing a patch finesse from one that has to be fixed can be tricky though (can \C{K3} be considered one-away, since \C{K2} is clued?), so the best option is if the player starting the finesse is the one immediately after the one that has to receive a fix clue; this way, in fact, no one can misplay after the first finessed card has been played, but before the fix clue is given.

\begin{remark}
	This technique works best if the card that requires a fix is a card that can end up being useful later, and especially if it is a critical card, as it would still most likely require a \emph{save clue} later on. Always keep in mind the card-per-clue ratio before giving any clue that requires a fix, and choose safer lines if the gain in efficiency is not clear.
\end{remark}

\subsection{Ambiguous finesse}

Sometimes a clue can be confusing, because multiple people will interpret it as a finesse on someone else. In these cases, it is better to have one of the involved players give the clue, to avoid ambiguities.

\begin{convention}[Passing a finesse]
	A player is only allowed to skip playing into a finesse for one of these reasons:
	
	\begin{itemize}
		\item they are playing a different, more important card (see Subsection~\ref{ssec:priorities});
		\item they are performing a finesse that no unoccupied player could have performed;
		\item they are preventing a dangerous discard that no one else could have prevented.
	\end{itemize}

	If a player is not playing into a finesse for a reason that is not specified above, then the finesse is probably an ambiguous finesse.
\end{convention}

\begin{example}	\hfill \\
	\label{ex:ambiguous-finesse}
	\begin{minipage}{0.45\textwidth}
		\begin{itemize}
			\item[\Large +] \CARD{R1} \CARD{Y2} \CARD{G5} \CARD{B1} \CARD{P2} \CARD{KX}
			\item[\Large A] \CARD{R4} \CARD{K3} \CARD{Y3} \CARD[n]{R5}
			\item[\Large B] \CARD{Y1} \CARD{Y2} \CARD[n]{K2} \CARD[n]{P5}
			\item[\Large C] \CARD{B2} \CARD{Y5} \CARD{P1} \CARD[n]{K5} 
			\item[\Large D] \CARD{B2} \CARD{G4} \CARD{Y4} \CARD{R1}
			\item[\Large E] \CARD[c]{P3} \CARD[c]{P4} \CARD{B3} \CARD{Y1}
		\end{itemize}
	\end{minipage}%
	\begin{minipage}{0.55\textwidth}
		\hfill \\
		
		\textbf{Description.} \\
		
		A clues \C{BX} to E, which is ambiguous as both C and D will think of it as a finesse on the other player. B clues \C{YX} to D as a finesse on A (their \C{Y3} is now in finesse position because of Convention~\ref{cyclic-rearrangement}), and C discards. The easier interpretation for D would be that C missed the finesse, but if they instead assume that everyone plays correctly, then they should play their \C{B2} instead.
		
		This is easier said than done: if there were no clues left, then C might have discarded as a save on D's chop; if C performed a finesse that D couldn't have possibly done themselves, then they might just think C wanted to gain efficiency and play their \C{B2} the next round. However, if it is clear that C wasn't otherwise busy, then D is supposed to play their \C{B2}.
		
		This would have been easier if A let C clue \C{BX} to E instead: this way, C's \C{B2} cannot possibly be involved.
	\end{minipage}
\end{example} \vspace{0.15 cm}

There are cases in which, even if one player understands an ambiguous finesse, they should still pass it back to the other player, to gain efficiency or to prevent a misplay.

\begin{example}	\hfill \\
	\begin{minipage}{0.45\textwidth}
		\begin{itemize}
			\item[\Large +] \CARD{R1} \CARD{Y2} \CARD{G5} \CARD{B1} \CARD{P2} \CARD{KX}
			\item[\Large A] \CARD{R4} \CARD{K3} \CARD{Y3} \CARD[n]{R5}
			\item[\Large B] \CARD{Y1} \CARD{Y2} \CARD[n]{K2} \CARD[n]{P5}
			\item[\Large C] \CARD{B2} \CARD{B3} \CARD{P1} \CARD[n]{K5} 
			\item[\Large D] \CARD{B2} \CARD{G4} \CARD{Y4} \CARD{R1}
			\item[\Large E] \CARD[c]{P3} \CARD[c]{P4} \CARD{B4} \CARD{Y1}
		\end{itemize}
	\end{minipage}%
	\begin{minipage}{0.55\textwidth}
		\hfill \\
		
		\textbf{Description.} \\
		
		Same as Example~\ref{ex:ambiguous-finesse}, but the finesse is now on a \C{B4} and C has a \C{B3} as well. Even if D understands that they have a \C{B2} in finesse position, they should still pass the finesse back to C, otherwise they would go on misplaying their \C{B2} as \C{B3}. After D passes too, C should understand that they have both \C{B2} and \C{B3} and play these cards in the subsequent turns.
		
		In this case A cannot leave the clue to anyone: C has no finesse to perform as they don't see any \C{B3}, and when D's turn comes, the cards in C's hand won't be in the appropriate position any more.
	\end{minipage}
\end{example} \vspace{0.15 cm}

\begin{example}	\hfill \\
	\label{ex:layered-ambiguous-finesse}
	\begin{minipage}{0.45\textwidth}
		\begin{itemize}
			\item[\Large +] \CARD{R1} \CARD{Y2} \CARD{G5} \CARD{B1} \CARD{P2} \CARD{KX}
			\item[\Large A] \CARD{R4} \CARD{K3} \CARD{Y3} \CARD[n]{R5}
			\item[\Large B] \CARD{Y1} \CARD{Y2} \CARD[n]{K2} \CARD[n]{P5}
			\item[\Large C] \CARD{R2} \CARD{B2} \CARD{P1} \CARD[n]{K5} 
			\item[\Large D] \CARD{B2} \CARD{G4} \CARD{Y4} \CARD{R1}
			\item[\Large E] \CARD[c]{P3} \CARD[c]{P4} \CARD{B3} \CARD{Y1}
		\end{itemize}
	\end{minipage}%
	\begin{minipage}{0.55\textwidth}
		\hfill \\
		
		\textbf{Description.} \\
		
		Same as Example~\ref{ex:ambiguous-finesse}, but C has a playable \C{R2} to the left of their \C{B2}. Once again, even if D understands that they have a \C{B2} in finesse position, they should still pass the finesse back to C, to get the \C{R2} played for free.
	\end{minipage}
\end{example} \vspace{0.15 cm}

Sometimes, one can have another player believe that a finesse is ambiguous to get some cards played for free.

\begin{convention}[Pass bluff]
	If a player is supposed to play into a finesse, but the next player has an unrelated playable card in finesse position for the finessing clue, they can \emph{pass} by discarding or giving a low-value clue to make the next player believe that the finesse was ambiguous and have them play. The player on whom the finesse was actually on \emph{must} play the next round, demonstrating that the finesse was on them, to prevent a misplay from the next player.
\end{convention}

\begin{example}	\hfill \\
	\begin{minipage}{0.45\textwidth}
		\begin{itemize}
			\item[\Large +] \CARD{R1} \CARD{Y2} \CARD{G5} \CARD{B1} \CARD{P2} \CARD{KX}
			\item[\Large A] \CARD{R4} \CARD{K3} \CARD{Y3} \CARD[n]{R5}
			\item[\Large B] \CARD{Y1} \CARD{Y2} \CARD[n]{K2} \CARD[n]{P5}
			\item[\Large C] \CARD{B2} \CARD{G1} \CARD{P1} \CARD[n]{K5} 
			\item[\Large D] \CARD{R2} \CARD{G4} \CARD{Y4} \CARD{R1}
			\item[\Large E] \CARD[c]{P3} \CARD[c]{P4} \CARD{B3} \CARD{Y1}
		\end{itemize}
	\end{minipage}%
	\begin{minipage}{0.55\textwidth}
		\hfill \\
		
		\textbf{Description.} \\
		
		Same as Example~\ref{ex:layered-ambiguous-finesse}, \C{R2} and \C{B2} are switched, and D has no \C{B2}. C is supposed to play their finessed \C{B2}, but they discard instead making D believe that they have a \C{B2} in finesse position. D sees that C didn't play, and goes on playing their \C{R2}. The next round, C \emph{must} play their \C{B2} to have D understand that they have been pass bluffed, otherwise D would play their \C{G4} as \C{B2}.
	\end{minipage}
\end{example} \vspace{0.15 cm}

Unlike a standard finesse, it is not possible to pass on a finesse that might be a bluff, because the other players won't see the ambiguity and will interpret the clue as a finesse on them instead.

\begin{convention}[Passing a bluff]
	A player is \emph{never} allowed to pass on a bluff. If they must, because they have to prevent a misplay or a critical discard, and the player performing the finesse is aware of that, then they are guaranteed that the clue was in fact a finesse and not a bluff.
\end{convention}

\begin{example}	\hfill \\
	\begin{minipage}{0.45\textwidth}
		\begin{itemize}
			\item[\Large +] \CARD{R1} \CARD{Y2} \CARD{G5} \CARD{B1} \CARD{P2} \CARD{KX}
			\item[\Large A] \CARD{R3} \CARD{K3} \CARD{Y3} \CARD[n]{R5}
			\item[\Large B] \CARD{Y1} \CARD{Y2} \CARD[n]{K2} \CARD[n]{P5}
			\item[\Large C] \CARD{B2} \CARD{G2} \CARD{P1} \CARD[n]{K5} 
			\item[\Large D] \CARD{G4} \CARD{R2} \CARD{Y4} \CARD{K4}
			\item[\Large E] \CARD[c]{P3} \CARD[c]{P4} \CARD{B3} \CARD{Y1}
		\end{itemize}
	\end{minipage}%
	\begin{minipage}{0.55\textwidth}
		\hfill \\
		
		\textbf{Description.} \\
		
		B clues \C{BX} to E as a finesse on C's \C{B2}. C sees that D has a critical \C{K4} in discard position, and also sees that they do not have anything urgent to do, so it is likely that D will discard the next turn. C has to prevent it, so they clue \C{RX} to D as a \emph{play clue}, keeping them busy. C knows that B knew that C had to prevent the \C{K4} discard, so they are promising that the \C{BX} clue to E is actually a finesse and not a bluff.
		
		If it were not, for example if C had a \C{Y3} in slot 1, then after C gives a clue to D, and D plays, E would think that they have a \C{B2} instead of a \C{B3}, as no one demonstrated the bluff, and would go on misplaying it. Thus B is not allowed to bluff here, and any potential bluff must be a finesse instead.
	\end{minipage}
\end{example} \vspace{0.15 cm}

Unfortunately, because of these issues, an ambiguous finesse is not always possible to perform.

\begin{example}	\hfill \\
	\begin{minipage}{0.45\textwidth}
		\begin{itemize}
			\item[\Large +] \CARD{R2} \CARD{YX} \CARD{G4} \CARD{BX} \CARD{P2} \CARD{KX}
			\item[\Large A] \CARD{B4} \CARD{Y5} \CARD{B4} \CARD[n]{K2}
			\item[\Large B] \CARD{B1} \CARD{R4} \CARD{G1} \CARD[n]{P5}
			\item[\Large C] \CARD{B2} \CARD{Y1} \CARD[n]{K3} \CARD[n]{K5} 
			\item[\Large D] \CARD{R2} \CARD{Y2} \CARD{R4} \CARD{R1}
			\item[\Large E] \CARD[c]{P3} \CARD[c]{P4} \CARD{B5} \CARD{Y1}
		\end{itemize}
	\end{minipage}%
	\begin{minipage}{0.55\textwidth}
		\hfill \\
		
		\textbf{Description.} \\
		
		Here, A might be tempted to clue \C{YX} to D, as a \emph{patch finesse} on B and C. However, if C's \C{X3} do not have contextual information by which it is not actually a \C{R3} (e.g. they got the \C{X3} clue as a save clue, and the \C{R3} is not critical), the clue doesn't work. In fact B thinks that the clue is either a regular finesse on them, or a patch finesse on C, and in either case they play their card in finesse position, which is a \C{B1}. Now, as B blind-played, rather than assuming it was a patch finesse on them, C will think that the clue is a patch finesse on B (as the \C{R4} is one-away) and thus they will misplay their \C{K3} as \C{R3}. These cards must be clued in some other way.
	\end{minipage}
\end{example} \vspace{0.15 cm}

\subsection{Combining finesse techniques}

Of course all these finesse techniques may be combined, but as always, be careful before doing something risky or unclear. Think to every possible interpretation of your clue, and if you're reasonably sure that your team mates will understand, then go for it. If not, better do something safer.