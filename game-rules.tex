\section{Game rules}

\emph{Hanabi} is a cooperative card game for 2-5 players. A standard Hanabi deck is composed of 50 cards, 10 for each of the colours \textcolor{red}{red} \red, \textcolor{yellow}{yellow} \yellow, \textcolor{green}{green} \green, \textcolor{blue}{blue} \blue, and \textcolor{purple}{purple} \purple, with values \one, \one, \one, \two, \two, \three, \three, \four, \four, \five.

There are also some alternative optional suits that can be used in place of the standard ones, or in addition to them, each with its own special properties:

\begin{itemize}
	\item \textcolor{teal}{teal} \teal, identical to the standard ones;
	\item black \black, composed of just one copy of each card (five in total);
	\item \textcolor{red}{r}\textcolor{orange}{a}\textcolor{yellow}{i}\textcolor{green}{n}\textcolor{blue}{b}\textcolor{violet}{o}\textcolor{purple}{w} \rainbow, with no own colour clue but touched by any other colour clue;
	\item \textcolor{gray}{white} \white, with no own colour clue and touched by no colour clue.
\end{itemize}

The latter two suits can also be played using just one copy of each card.


\subsection{Setup}

In order to play a game, the players choose up to six suits to use during the game. The website \href{http:/hanabi.live}{hanabi.live} offers these and many other variants, but in this document we will only discuss the ones listed above,  using \emph{Black (6 suits)} for five players as default.

At the beginning of the game, each player draws either 5 (for 2 or 3 players) or 4 cards (for 4 or 5 players). The cards must face \emph{the other players}: one player does not know which cards are in their hands, but they know which cards are in the other players hands.

\begin{definition}
	The \emph{board} of the game is the set of the highest cards played for each suit (it is \C{X0} in a suit if no card of a that suit has been played).
\end{definition}

The players start with 8 \emph{clues} and 3 \emph{lives}. The starting board is \[ \text{\CARD{RX} \CARD{YX} \CARD{GX} \CARD{BX} \CARD{PX}  \CARD{KX}} \] where the \C{X0} number next to the symbol is omitted for empty piles.

\begin{definition}
	A card is \emph{immediately playable} if it is the successor (i.e. it has the same colour and the number is one unit higher) of a card that belongs to the current configuration.
\end{definition}

The starting player is selected at random. The game then proceeds clockwise.

\subsection{Game turn}

During their turn, the current player must do one of the following actions.

\paragraph{Play a card} The current player picks a card in their hand and puts it face-up on the table. If the card is immediately playable, then it replaces the current one of the same suit in the configuration. Furthermore, if the card is a \C{X5}, the players gain one clue (unless they have 8).

If the card is not immediately playable, then it is discarded and the players lose a life. In either case, the current player draws another card (if possible).

\paragraph{Discard a card} The current player picks a card in their hand and puts it in the \emph{discard pile}. That card cannot be used any more during the game. The players gain a clue (unless they have 8), and the current player draws another card (if possible).

\paragraph{Give a clue} There must be at least one clue to perform this action. The current player chooses another player and tells them which of their cards have a certain colour (e.g. \emph{these cards are \C{BX}}), or which ones have a certain value (e.g. \emph{these cards are \C{X3}}). They must point exactly the subset of the other player's hand with that property, and that subset cannot be empty. The players lose a clue.

\subsection{Game end}

The game ends immediately if the players have 0 lives. If that does not happen, after the last card of the deck is drawn, each player has exactly one turn left (including the one who drew the last card). The \emph{final score} is the number of cards played when the game ends, or equivalently the sum of the values in the final configuration.

%%\begin{note}
%%	If the players score 30 with 3 lives left, then the score is 30L, because why not make it like an exam?
%%\end{note}
%
%In our version of the game, the players may always look at the discard pile, and they may always ask which clues have been given so far, by who, and when. They are also allowed to keep track of the clues that were given by using any kind of mnemonic aid. This is not supposed to be a game based on memory: public information is always available. Finally, the players are allowed to rearrange their hand \emph{algorithmically} after any event that conventionally triggers a rearrangement, but they may not rearrange their hand in any other moment (that would be cheating). Each player's current hand ordering is known to everyone.