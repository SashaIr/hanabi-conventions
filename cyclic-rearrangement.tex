\section{Cyclic rearrangement}

It happens quite often that a player discards to prevent the next player from doing the same, usually because the next player's chop card is important. It also may happen that a player has two cards of the same colour that they should play, but the higher is on the left. Both issues can be fixed with this idea.

\begin{convention}[Cyclic rearrangement]
	\label{cyclic-rearrangement}
	When a player gives a clue, they should move their chop card to slot 1. If all their cards are clued, then they do nothing.
\end{convention}

The cyclic rearrangement works very well with the following convention.

\begin{convention}[Chop focus]
	\label{chop-focus}
	When a player receives a \emph{play clue} on a set of cards that includes their chop, then that card is the one which should be played first. The other cards are played left to right as usual.
\end{convention}

The reason behind this is the fact that sometimes a player has multiple playable cards of the same colour, and if the first to be played is on chop, then the other players might be tempted to wait until the player gives a clue and cyclically rearranges their hand. This is risky, because it lets a playable card stay on chop for at least a round. Convention~\ref{chop-focus} solves this issue. Of course the drawback is that configurations in which a non playable card of the same colour is on chop cannot be clued, but this is less of an issue because if the player gives a clue themselves, then the non playable card moves and the playable one gets closer to the chop (where it can be clued); if they discard instead, the lost card is most likely not as valuable, because it wasn't playable.