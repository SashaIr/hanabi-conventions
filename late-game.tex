\section{Late game conventions}
\label{sec:late-game}

When the game is about to end, efficiency is often a lot less useful than tempo, and discarding should be modulated among players. Precise computing of game lines is required, and more often than not it is better to just give low value clues to avoid discarding. 

\subsection{Positional clues}

It can happen that a player knows, by negative information, that all their cards are trash, and can thus choose which one to discard if they have to.

\begin{convention}[Positional trash discard]
	If a player knows that they do not have any relevant card in their hand, they can discard from any slot to communicate to the next unoccupied player to play the card in the matching slot.
\end{convention}

\begin{example}	\hfill \\
	\begin{minipage}{0.45\textwidth}
		\begin{itemize}
			\item[\Large +] \CARD{R5} \CARD{Y3} \CARD{G5} \CARD{B3} \CARD{P5} \CARD{K4}
			\item[\Large A] \CARD{R3} \CARD{G1} \CARD{R2} \CARD{B2}
			\item[\Large B] \CARD{Y1} \CARD{Y5} \CARD{Y4} \CARD{P4}
			\item[\Large C] \CARD[c]{B4} \CARD{Y1} \CARD{G3} \CARD{B3}
			\item[\Large D] \CARD[n]{B5} \CARD{B1} \CARD{G4} \CARD{P1}
			\item[\Large E] \CARD[c]{K5} \CARD{P2} \CARD{R1} \CARD{G3}
		\end{itemize}
	\end{minipage}%
	\begin{minipage}{0.55\textwidth}
		\hfill \\
		
		\textbf{Description.} \\
		
		A to play, no clues left, two cards in the deck. One \C{Y4} has been discarded. In order to achieve a perfect score, B must play their \C{Y4} immediately, but they don't know which card is it (from B's perspective, it may still be in the deck). All the other players already have a play.
		
		A knows that they don't have any relevant card, so they discard their \C{R2} from slot 3, asking B (the first player who doesn't know what to do) to play their card in the same position (the \C{Y4}). They do, then C, D, E play their clued cards, A clues \C{X5} to B with the clue they gained the last round, B plays the \C{Y5}, and the players score a 30.
	\end{minipage}
\end{example} \vspace{0.15 cm}

Sometimes it can be worth it to perform a positional trash discard even if the discarded card is potentially relevant.

\begin{example}	\hfill \\
	\begin{minipage}{0.45\textwidth}
		\begin{itemize}
			\item[\Large +] \CARD{R4} \CARD{Y3} \CARD{G5} \CARD{B3} \CARD{P5} \CARD{K4}
			\item[\Large A] \CARD{R3} \CARD{K5} \CARD{R2} \CARD{B2}
			\item[\Large B] \CARD[n]{B5} \CARD{B1} \CARD{G4} \CARD{P1}
			\item[\Large C] \CARD{P4} \CARD{Y5} \CARD{Y4} \CARD{Y1}
			\item[\Large D] \CARD[cn]{B4} \CARD{Y1} \CARD{G3} \CARD{B3}
			\item[\Large E] \CARD{P2} \CARD{B1} \CARD{R1} \CARD{G3}
		\end{itemize}
	\end{minipage}%
	\begin{minipage}{0.55\textwidth}
		\hfill \\
		
		\textbf{Description.} \\
		
		As before, A to play, no clues left, two cards in the deck, one \C{Y4} has been discarded. B, C, and D have been permuted, and also A, not E, has the \C{K5} in their hand. Notice that there is no way to point the \C{Y4} in C's hand with one single clue.
			
		A still discards their \C{R2} from slot 3. It may have been the \C{K5}, but even then this is the correct play. B knows that their only relevant card is the \C{B5}, so in particular they know that they must not play. Also, C has to play two cards, so B must not discard either. In this case, B clues \C{X5} to A. C plays the \C{Y4} according to A's discard, then D plays their \C{B4}, and E, with no clues left, discards their \C{R1} from slot 3 to communicate to C that their \C{Y5} is in that slot (they drew a new card after playing their \C{Y4}). A and B play their \C{X5}'s, C play their \C{Y5} because of E's positional trash discard, and once again the players score a 30.
	\end{minipage}
\end{example} \vspace{0.15 cm}

Unfortunately, if it is not globally known that a player knows that they do not have any relevant card, and the card to be played in the next unoccupied player's hand is in the same slot as the current player's chop, then a positional trash discard will be interpreted as a regular discard and will not trigger a play.

\begin{convention}[Positional chop misplay]
	If a player knows that they do not have any relevant card in their hand, and the next unoccupied player has a playable card in the slot matching the current player's chop, and there will still be cards left in the deck when the next unoccupied player's turn comes, then they can purposefully misplay their chop card to communicate that player to play the card in that slot.
\end{convention}

\subsection{Stall clues}

In the late game, it often happens that a player is not supposed to discard, but any clue that they can give is potentially misinterpreted as a \emph{play clue} or a \emph{finesse}. This is inconvenient, so during the late game it is better to prevent that from happening.

\begin{convention}[Late game stall numbers]
	In the late game, a number clue on any relevant card that could have been given as a colour clue is a \emph{stall clue}, which is just passing some information about that card and it is not supposed to trigger any finesse or being interpreted as \emph{play clue}.
\end{convention}

This usually works well as it rarely happens that there is more than one relevant card with a given number that is not visible in any other player's hand, which means that such a clue often gives complete information on the identity of that card (and can work as a \emph{play clue} if the only possible option for that card is actually playable). It is also one of the most efficient ways to spend a clue in a turn in which a player is not supposed to discard.